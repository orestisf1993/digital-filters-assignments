\section{Ερώτημα Γ}\label{section:C}
Οι υπολογισμοί έγιναν στη συνέχεια του αρχείου \texttt{main1.m}.
Η υλοποίηση του αλγορίθμου Levinson-Durbin γίνεται στο αρχείο \texttt{my_levinson.m}.
Προκύπτουν οι εξής διαφορές:
\begin{enumerate}
\item Διαφορά (MSE) στους συντελεστές πρόβλεψης $\alpha_m$ σε σχέση με τον αλγόριθμο \mintinline{MATLAB}!levinson! του MATLAB: \mintinline{MATLAB}!9.833868e-30!.
\item Διαφορά (MSE) στους παραμέτρους ανάκλασης $\Gamma_m$ σε σχέση με τον αλγόριθμο \mintinline{MATLAB}!levinson! του MATLAB: \mintinline{MATLAB}!9.223482e-30!.
\item Διαφορά στη δύναμη του σφάλματος πρόβλεψης $P_m$ σε σχέση με τον αλγόριθμο \mintinline{MATLAB}!levinson! του MATLAB: \mintinline{MATLAB}!9.984021e-31!.
\item Διαφορά στη σύγκριση των $L^T \underbar{\gamma}$ και $\underbar{\wienero}$: \mintinline{MATLAB}!7.967151e-04!.
\end{enumerate}

\begin{figure}
\includegraphics[width=\linewidth]{plots/wiener}
\caption{Μέσο τετραγωνικό σφάλμα εξόδου φίλτρου για τους συντελεστές $\wienero$}
\end{figure}
\begin{figure}
\includegraphics[width=\linewidth]{plots/jpe}
\caption{Μέσο τετραγωνικό σφάλμα εξόδου φίλτρου για τον estimator}
\end{figure}
\FloatBarrier