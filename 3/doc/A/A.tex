\section{Ερώτημα Α}\label{section:A}
Σε διάφορες περιπτώσεις ένα ευρυζωνικό σήμα καταστρέφεται από περιοδικές παρεμβολές και δεν υπάρχει κανένα σήμα αναφοράς.
Η διάταξη του \imagerefc{init-system}{Σχήματος \ref{fig:init-system}}
μπορεί να χρησιμοποιηθεί για να απομακρύνει περιοδικές παρεμβολές που
αλλοιώνουν ένα ευρυζωνικό σήμα (broadband signal), χωρίς να απαιτεί τη χρήση σήματος αναφοράς.

Το σύνθετο σήμα $s(n)$ περιέχει το σήμα $x(n)$ και τον λευκό θόρυβο $v(n)$.
Ένα κανάλι του $s(n)$ στέλνεται στη διάταξη $Z^{-\Delta}$ που επιβάλει σταθερή καθυστέρηση $\Delta$, αρκετά μεγάλη ώστε να προκαλέσει την αποσυσχέτηση του $x(n)$ με το $v(n)$.
Το καθυστερημένο σήμα $u(n)$ φιλτράρεται μέσω του φίλτρου Wiener $\wienero$ που μας δίνει το σήμα $y(n)$ στο οποίο εμφανίζεται η περιοδική παρεμβολή. Όταν αφαιρέσουμε το $y(n)$ από το αρχικό σήμα $s(n)$ παίρνουμε το καθαρό σήμα $e(n)$.
