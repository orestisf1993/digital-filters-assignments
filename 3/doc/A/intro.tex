\newcommand{\weiner}[1]{\mathbf{w}\left(#1\right)}
\newcommand{\wienero}{\mathbf{w_o}}
\newcommand{\Rmatrix}{\mathbf{R}}
\newcommand{\Pvector}{\mathbf{p}}
\section{Εισαγωγή}
\imagehere{init-system}{Το σύστημά μας}
Μας δίνεται το σύστημα του \imagerefc{init-system}{Σχήματος \ref{fig:init-system}}
που περιγράφεται από τις εξής εξισώσεις:
\begin{align}
x(n) &= A\left(\sin\left(2 \pi f_o n + \phi\right) + \cos\left(4 \pi f_o n + \phi\right) + \cos\left(7 \pi n + \frac{\phi}{3} \right)\right), f_o=\frac{1}{4}, \phi=\frac{\pi}{2}, A=2.3\label{eq:x}\\
s(n) &= x(n) + v(n)\label{eq:s}\\
u(n) &= s(n - \Delta)\label{eq:u}\\
d(n) &= s(n)\\
e(n) &= d(n) - y(n)\label{eq:e}
\end{align}
Όπου το $v(n)$ είναι λευκός θόρυβος με μηδενική μέση τιμή και διακύμανση $\sigma_{v}^2 = 0.34$
και $\Delta = 10$.
