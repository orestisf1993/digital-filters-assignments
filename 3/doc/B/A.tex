\section{Ερώτημα Α}
\subsection{Υπολογισμός πίνακα αυτοσυσχέτισης}
Από την \eqref{eq:u} έχουμε:
\begin{align*}
\begin{split}
E[u(n) u^*(n-k)]={}&
-0.87E[u(n-1) u^*(n-k)]\\
&-0.12E[u(n-2) u^*(n-k)]\\
&-0.032E[u(n-3) u^*(n-k)]\\
&+E[v_1(n) u^*(n-k)]\\
\end{split}\\
\implies{}&
r_{uu}(k)
+0.87r_{uu}(k-1)
+0.12r_{uu}(k-2)
+0.032r_{uu}(k-3)
= r_{{v_1}u}(k)\numberthis
\end{align*}
όμως
\begin{equation}
r_{{v_1}u}(k) = \left\{
    \begin{array}{lr}
        \sigma_{v_1}^2,& \text{αν }k = 0\\
        0,& \text{αν }k \neq 0
    \end{array}
\right\}
\end{equation}
άρα
\begin{align*}
k = 0 &\implies r_{uu}(0) + 0.87r_{uu}(-1) + 0.12r_{uu}(-1) + 0.032r_{uu}(-3) = r_{{v_1}u}(0) = \sigma_{v_1}^2 = 0.73\\
k = 1 &\implies r_{uu}(1) + 0.87r_{uu}(0) + 0.12r_{uu}(0) + 0.032r_{uu}(-2) = r_{{v_1}u}(1) = 0\\
k = 2 &\implies r_{uu}(2) + 0.87r_{uu}(1) + 0.12r_{uu}(1) + 0.032r_{uu}(-1) = r_{{v_1}u}(2) = 0\\
k = 3 &\implies r_{uu}(3) + 0.87r_{uu}(2) + 0.12r_{uu}(2) + 0.032r_{uu}(0) = r_{{v_1}u}(3) = 0
\end{align*}
Λύνουμε το σύστημα στο MATLAB και βρίσκουμε τον Toeplitz πίνακα του αποτελέσματος ($\Rmatrix$):
\begin{code}
\begin{minted}{MATLAB}
>> A = [1 0.87 0.12 0.032; 0.87 1.12 0.032 0; 0.12 0.902 1 0; 0.032 0.12 0.87 1];
>> B = [0.73; 0; 0; 0];
>> r = A \ B

r =

    1.9927
   -1.5818
    1.1877
   -0.9072

>> R = toeplitz(r)

R =

    1.9927   -1.5818    1.1877   -0.9072
   -1.5818    1.9927   -1.5818    1.1877
    1.1877   -1.5818    1.9927   -1.5818
   -0.9072    1.1877   -1.5818    1.9927
\end{minted}
\caption{Επίλυση συστήματος στο MATLAB και εύρεση του $\Rmatrix$}
\end{code}

\subsection{Υπολογισμός διανύσματος ετεροσυσχέτισης}
Από τις \eqref{eq:u} και \eqref{eq:d}:
\begin{dmath}
p(-k) = E[u(n - k) d^*(n)] = \ldots =
-0.23r_{uu}(k)+0.67r_{uu}(k+1)-0.18r_{uu}(k+2)+0.39r_{uu}(k+3)
\end{dmath}
καθώς τα $u$ και $x$ είναι ασυσχέτιστα.

Άρα, μπορούμε να βρούμε το $\Pvector$:
\begin{align*}
k = 0 \implies& p(0)  = -0.23r_{uu}(0)+0.67r_{uu}(1)-0.18r_{uu}(2)+0.39r_{uu}(3) = -0.7702\\
k = 1 \implies& p(-1) = -0.23r_{uu}(1)+0.67r_{uu}(2)-0.18r_{uu}(3)+0.39r_{uu}(4) = 2.4469\\
k = 2 \implies& p(-2) = -0.23r_{uu}(2)+0.67r_{uu}(3)-0.18r_{uu}(4)+0.39r_{uu}(5) = -2.3086\\
k = 3 \implies& p(-3) = -0.23r_{uu}(3)+0.67r_{uu}(4)-0.18r_{uu}(5)+0.39r_{uu}(6) = 2.0663
\end{align*}
και
\begin{equation}
\Pvector = \begin{bmatrix} p(0) & p(-1) & p(-2) & p(-3)\end{bmatrix}^T = \begin{bmatrix} -0.7702 & 2.4469 & -2.3086 & 2.0663 \end{bmatrix}^T
\end{equation}

\subsection{Υπολογισμός βέλτιστων συντελεστών φίλτρου wiener}
Σύμφωνα με την εξίσωση Wiener-Hopf έχουμε:
\begin{equation}
\Rmatrix w_o = \Pvector \implies w_o = \Rmatrix^{-1}\Pvector
\end{equation}
\begin{code}
\begin{minted}{MATLAB}
>> P = [-0.7702; 2.4469; -2.3086; 2.0663];
>> R \ P

ans =

    1.5722
    2.2379
    0.0363
    0.4477
\end{minted}
\caption{Επίλυση εξίσωσης Wiener-Hopf στο MATLAB}
\end{code}
Άρα
\begin{equation}
w_o = \begin{bmatrix} 1.5722 & 2.2379 & 0.0363 & 0.4477 \end{bmatrix}^T
\end{equation}
