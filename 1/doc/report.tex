%&../../preamble
% polyglossia
\usepackage{polyglossia}
\setmainlanguage{greek}
\setotherlanguages{english}

% Fonts
% fonts can't go in the .fmt file
\usepackage{fontspec}
\setmainfont[Mapping=tex-text]{DejaVu Sans}
\newfontfamily\greekfont[Script=Greek]{DejaVu Sans}
\newfontfamily\greekfontsf[Script=Greek]{DejaVu Sans}
\setmonofont{Hack}
\newfontfamily\greekfonttt[Scale=1.0]{Hack}
\usepackage{microtype} % microtype is font-dependant

\author{%
  Φλώρος-Μαλιβίτσης Ορέστης, 7796 \href{mailto:orestisf@ece.auth.gr}{orestisf@ece.auth.gr}\\
  \textbf{Τομέας Ηλεκτρονικής}\\
  \textbf{Τμήμα Ηλ. Μηχανικών / Μηχανικών ΗΥ}\\
  \textbf{Αριστοτέλειο Πανεπιστήμιο Θεσσαλονίκης}
}
\titlepic{\includegraphics[width=0.40\textwidth]{../../university}}

% for code with higlighting
\usepackage{listings}
\usepackage{lstautogobble}
\usepackage{color}
\definecolor{mygreen}{rgb}{0,0.6,0}
\definecolor{mygray}{rgb}{0.5,0.5,0.5}
\definecolor{mymauve}{rgb}{0.58,0,0.82}
\lstset{ %
    backgroundcolor=\color{white},   % choose the background color; you must add \usepackage{color} or \usepackage{xcolor}
    basicstyle=\normalsize\ttfamily, % the size of the fonts that are used for the code
    breakatwhitespace=false,         % sets if automatic breaks should only happen at whitespace
    breaklines=false,                % sets automatic line breaking
    captionpos=t,                    % sets the caption-position to top
    commentstyle=\color{mygreen},    % comment style
    escapeinside={\%*}{*)},          % if you want to add LaTeX within your code
    extendedchars=true,              % lets you use non-ASCII characters; for 8-bits encodings only, does not work with UTF-8
    frame=single,                    % adds a frame around the code
    keepspaces=true,                 % keeps spaces in text, useful for keeping indentation of code (possibly needs columns=flexible)
    keywordstyle=\color{blue},       % keyword style
    language=Octave,                    % the language of the code
    numbers=left,                    % where to put the line-numbers; possible values are (none, left, right)
    numbersep=5pt,                   % how far the line-numbers are from the code
    numberstyle=\tiny\color{mygray}, % the style that is used for the line-numbers
    rulecolor=\color{black},         % if not set, the frame-color may be changed on line-breaks within not-black text (e.g. comments (green here))
    showspaces=false,                % show spaces everywhere adding particular underscores; it overrides 'showstringspaces'
    showstringspaces=false,          % underline spaces within strings only
    showtabs=false,                  % show tabs within strings adding particular underscores
    stepnumber=1,                    % the step between two line-numbers. If it's 1, each line will be numbered
    stringstyle=\color{mymauve},     % string literal style
    tabsize=4,                       % sets default tabsize to 2 spaces
    title=\lstname                   % show the filename of files included with \lstinputlisting; also try caption instead of title
}


\renewcommand{\imageref}[1] {%
\hyperref[fig:#1]{Σχήμα \ref{fig:#1}}}
\renewcommand{\imagerefc}[2] {%
\hyperref[fig:#1]{#2}}

\newcommand{\imagehere}[2]{%
\begin{figure}[H]%
\centering%
\includegraphics[keepaspectratio, width=1.0\linewidth]{images/#1}%
\caption{#2}
\label{fig:#1}%
\end{figure}%
}

\title{1η Εργασία στα Ψηφιακά Φίλτρα}
\author{Ορέστης Φλώρος-Μαλιβίτσης, 7796\\
Τομέας Ηλεκτρονικής,\\
Τμήμα Ηλ. Μηχανικών / Μηχανικών ΗΥ,\\
Αριστοτέλειο Πανεπιστήμιο Θεσσαλονίκης}
\titlepic{\includegraphics[width=0.40\textwidth]{../../university}}

\setcounter{section}{-1} % start sections from 0.
\begin{document}
\deactivateBG
\maketitle
\tableofcontents\newpage
%\lstlistoflistings
%\listoffigures
%\listoftables
\section{Εισαγωγή}
\imagehere{init-system}{Το σύστημά μας}
Μας δίνεται το σύστημα του \imagerefc{init-system}{Σχήματος \ref{fig:init-system}}
για το οποίο ισχύει:
\begin{equation}
u(n) = -0.18 u(n-1) + v(n)\label{eq:u}
\end{equation}

Το $H(z)$ πρόκειται για ένα γραμμικό σύστημα το οποίο υλοποιείται στο αρχείο MATLAB \texttt{plant.p}.

\section{Ερώτημα Α}
\subsection{Υπολογισμός πίνακα αυτοσυσχέτισης $\Rmatrix$}
Από την \eqref{eq:u} έχουμε:
\begin{align*}
&E[u(n) u^*(n-k)] = -0.78E[u(n-1) u^*(n-k)] + E[v(n) u^*(n-k)]\\
\implies{}&r_{uu}(k) + 0.78r_{uu}(k-1) = r_{vu}(k)
\numberthis
\end{align*}
όμως
\begin{equation}
r_{vu}(k) = \left\{
    \begin{array}{lr}
        \sigma_v^2,& \text{αν }k = 0\\
        0,& \text{αν }k \neq 0
    \end{array}
\right\}
\end{equation}
άρα
\begin{alignat}{3}
k = 0 &\implies r_{uu}(0) + 0.78r_{uu}(-1) = r_{vu}(0) &\implies r_{uu}(0) + 0.78r_{uu}(1) &= \sigma_v^2 = 0.19\\
k = 1 &\implies r_{uu}(1) + 0.78r_{uu}(0) = r_{vu}(1) &\implies r_{uu}(1) + 0.78r_{uu}(0) &= 0
\end{alignat}
Λύνουμε το σύστημα στο MATLAB:
\begin{lstlisting}[caption={Επίλυση συστήματος στο MATLAB}]
>> A = [1 0.78; 0.78 1];
>> B = [0.19; 0];
>> A \ B

ans =

    0.4852
   -0.3784
\end{lstlisting}
Έτσι, ο πίνακας αυτοσυσχέτισης $\Rmatrix$ είναι:
\begin{equation}
\Rmatrix = \begin{bmatrix}
    r_{uu}(0) & r_{uu}(1)\\
    r_{uu}(1) & r_{uu}(0)
\end{bmatrix} =
\begin{bmatrix}
    0.4852 & -0.3784\\
    -0.3784 & 0.4852
\end{bmatrix}
\end{equation}

\subsection{Υπολογισμός διανύσματος ετεροσυσχέτισης $\Pvector$}
Από τις \eqref{eq:u} και \eqref{eq:x+v}:
\begin{align*}
p(-k) ={}& E[u(n - k) d^*(n)]\\
={}& E[(u(n - k - 1) + v(n - k)) (x^*(n) + v^*(n))]\\
={}& -0.78E[u(n - k - 1) x^*(n)]\\
& -0.78E[u(n - k - 1) v^*(n)]\\
& + E[v(n - k) x^*(n)]\\
& + E[v(n - k) v^*(n)]
\numberthis\label{eq:p-init}
\end{align*}
Όμως επειδή το $x$ είναι ασυσχέτιστο με τα $v$ και $u$ ισχύει:
\begin{align*}
E[u(n - k - 1) x^*(n)] &= 0\\
E[v(n - k) x^*(n)] &= 0
\end{align*}
η \eqref{eq:p-init} γίνεται:
\begin{equation}
p(-k) = -0.78E[u(n - k - 1) v^*(n)] + E[v(n - k) v^*(n)] = -0.78r_{uv}(k + 1) + r_{vv}(k)
\end{equation}
Άρα, μπορούμε να βρούμε το $\Pvector$:
\begin{align*}
k = 0 \implies& p(0) = -0.78r_{uv}(1) + r_{vv}(0) = 0.19\numberthis\\
k = 1 \implies& p(-1) = -0.78r_{uv}(2) + r_{vv}(1) = 0\numberthis
\end{align*}
και
\begin{equation}
\Pvector = \begin{bmatrix} p(0) & p(-1)\end{bmatrix}^T = \begin{bmatrix} 0.19 & 0\end{bmatrix}^T
\end{equation}

\subsection{Υπολογισμός βέλτιστων συντελεστών φίλτρου wiener $w_o$}
Σύμφωνα με την εξίσωση Wiener-Hopf έχουμε:
\begin{equation}
\Rmatrix w_o = \Pvector \implies w_o = \Rmatrix^{-1}\Pvector
\end{equation}
\begin{lstlisting}
>> r = A \ B;
>> R = [r(1) r(2); r(2) r(1)];
>> P = [0.19 0]';
>> R \ P

ans =

    1.0000
    0.7800
\end{lstlisting}
Άρα
\begin{equation}
w_o = \begin{bmatrix} 1 & 0.78\end{bmatrix}^T
\end{equation}

\section{Ερώτημα Β}
Η αναδρομική συνάρτηση για τον υπολογισμό του FFT βρίσκεται στο αρχείο \texttt{fft/fftrecursive.m}.
\begin{code}
\inputminted[frame=single, breaklines=true, linenos=true, firstline=11, lastline=19]{MATLAB}{../fft/fftrecursive.m}
\caption{Το κυρίως μέρος της \texttt{fftrecursive}}
\end{code}

Όπως φαίνεται και από τα σχόλια το κόστος είναι:
\begin{equation}
T(n) = T(n/2) + T(n/2) + 6n/2 + 2n = 2T(\frac{n}{2}) + 5n
\end{equation}

\begin{itemize}
\item $T(1)=0$ αφού δεν γίνεται καμία αριθμητική πράξη.
\item $T(2)=4$ αφού:
\begin{itemize}
\item $2T(n/2)=2T(1)=0$.
\item \mintinline{MATLAB}!d = exp(-2 * pi * 1i / 2) .^ (0:2 / 2 - 1) = 1!
\item \mintinline{MATLAB}!z = (d.') .* y_bottom = y_bottom;! 0 κόστος.
\item \mintinline{MATLAB}!y = [y_top + z; y_top - z];! 2 προσθέσεις μιγαδικών άρα 4 flops.
\end{itemize}
\item $T(4)=2T(2)+5\cdot4$.
\item $T(8)=2T(4)+5\cdot8=4T(2)+5\cdot8+5\cdot8=4T(2)+2\cdot5\cdot8$.
\item $T(16)=2T(8)+5\cdot16=8T(2)+2\cdot5\cdot16+5\cdot16=8T(2)+3\cdot5\cdot16$.
\\\vdots
\item $T(n)=\frac{n}{2}T(2)+\log_2{\frac{n}{2}}\cdot5n$.
\item $T(2n)=2T(2n/2)+5\cdot2n=
2(\frac{n}{2}T(2)+\log_2{\frac{n}{2}}\cdot5n)+5\cdot2n=
\frac{2n}{2}T(2)+\log_2{\frac{n}{2}}\cdot10n + 10n=
\frac{2n}{2}T(2)+(\log_2{n}-\log_2{2}+1)\cdot10n=
\frac{2n}{2}T(2)+\log_2{\frac{2n}{2}}\cdot5(2n)$.
\end{itemize}

Άρα σύμφωνα με την μαθηματική επαγωγή:
\begin{equation}
T(n)=\frac{n}{2}T(2)+\log_2{\frac{n}{2}}\cdot5n\in O(n\log_2{n})
\end{equation}
Επίσης, αφού $T(2)=4$, επιβεβαιώνεται και η σχέση της εκφώνησης:
\begin{equation}
T(n)=\frac{n}{2}T(2)+6\frac{n}{2}+2T(\frac{n}{2})=4\frac{n}{2}+3n+2T(\frac{n}{2})=2T(\frac{n}{2}) + 5n
\end{equation}

\newcommand{\dplothere}[2]{%
\begin{figure}[htb]
\centering
\begin{subfigure}{\linewidth}
\centering
    \includegraphics[height=0.48\textheight]{plots/#1-1}
\end{subfigure}
\par\bigskip % force a bit of vertical whitespace
\begin{subfigure}{\linewidth}
\centering
    \includegraphics[height=0.48\textheight]{plots/#1-2}
\end{subfigure}
\caption{#2}
\label{fig:#1}%
\end{figure}
}
\section{Ερώτημα Γ}
Υλοποιήθηκαν οι εξής εκδοχές του αλγορίθμου \emph{Block LMS}:
\begin{enumerate}[a)]
\item \label{item:simple}\texttt{lms/blocklms\_simple.m}: Με δύο εμφωλευμένους βρόχους.
\item \label{item:array}\texttt{lms/blocklms\_array.m}: Με ένα βρόχο και πράξεις πινάκων.
\item \label{item:fft}\texttt{lms/blocklms\_fft.m}: Με προσαρμογή στο πεδίο της συχνότητας κάνοντας χρήση του FFT.
\item \label{item:ufft}\texttt{lms/blocklms\_unconstrained\_fft.m}: Με μη περιορισμένη (unconstrained) προσαρμογή στο πεδίο της συχνότητας.
\end{enumerate}

Η πιο συμφέρουσα υπολογιστικά υλοποίηση είναι αυτή με τη
\hyperref[item:ufft]{μη περιορισμένη προσαρμογή στο πεδίο της συχνότητας}.
Το σχετικό διάγραμμα φαίνεται στο \imageref{performance}.
Στα αρχεία \texttt{profiler\_results/file*.html} φαίνεται αναλυτικά ο χρόνος εκτέλεσης για κάθε συνάρτηση για \mintinline{MATLAB}!n = 45000!.
\dplothere{performance}{Ταχύτητα εκτέλεσης των 4 αλγορίθμων LMS}

Τέλος, οι καμπύλες εκμάθησης για τους αλγορίθμους φαίνονται στο \imageref{learning-curve}.
Παρατηρούμε ότι για τους 3 πρώτους αλγορίθμους οι καμπύλες εκμάθησης είναι ίδιες αφού οι αλγόριθμοι είναι ισοδύναμοι.
\begin{figure}
\includegraphics[width=\linewidth]{plots/learning-curve}
\caption{Καμπύλες εκμάθησης των 4 αλγορίθμων LMS}
\label{fig:learning-curve}
\end{figure}

\section{Ερώτημα Δ}
Στο αρχείο \url{matlab/wiener.m} βρίσκεται ένα MATLAB script που εκτελεί τις εξής λειτουργίες:
\begin{itemize}
\item Φόρτωση των αρχείων \url{matlab/sound.mat} και \url{matlab/noise.mat}.
\item Υπολογισμός των πινάκων $\Rmatrix_u$ και $\Pvector$.
\begin{lstlisting}[style=chunk]
%% Calculate needed values.
% Αutocorrelation sequence.
ruu = xcorr(u, max_lag, 'biased');
% Construct autocorrelation matrix.
R = toeplitz(ruu(n_coeffs: -1:1));
% Cross-correlation sequence.
rdu = xcorr(d, u, max_lag, 'biased');
% Cross-correlation vector.
p = rdu(n_coeffs:2 * n_coeffs - 1);
\end{lstlisting}
\item Υπολογισμός των συντελεστών του φίλτρου wiener.
Η διαδικασία γίνεται είτε με την επίλυση της εξίσωσης Wiener-Hopf
\begin{lstlisting}[style=chunk]
w_opt = R \ p; % Wiener-Hopf solution
\end{lstlisting}
είτε με τη χρήση του steepest descent.
\begin{lstlisting}[style=chunk]
[~, w_opt] = steepest_descent(p, R, mu, 0.0000001, 100000);
\end{lstlisting}
\item Φιλτράρισμα του ήχου.
\begin{lstlisting}[style=chunk]
%% Filter sound.
y = conv(w_opt, u);
e = d - y(1:length(d));
\end{lstlisting}
\item Αναπαραγωγή και αποθήκευσή του.
\begin{lstlisting}[style=chunk]
%% Play result.
sound(e, Fs);
audiowrite('result.wav', e, Fs);
\end{lstlisting}
\end{itemize}
Μετά την αναπαραγωγή του φιλτραρισμένου ήχου αναγνωρίστηκε το τραγούδι \href{https://youtu.be/xSHYlSxQyJM}{Bang Bang} εκτέλεση (διασκεύη) από τη Nancy Sinatra. (Μπορεί να είναι μια έκδοση ελαφρά διαφορετική από το παραπάνω link)

\end{document}
