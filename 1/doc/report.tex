%&../../preamble
% automatic hyphenation for 2 languages
% http://www.mechpedia.gr/wiki/Hyphenation_-_%CE%A5%CF%86%CE%B5%CE%BD%CF%8E%CF%83%CE%B5%CE%B9%CF%82#.CE.91.CF.85.CF.84.CF.8C.CE.BC.CE.B1.CF.84.CE.B5.CF.82_.CF.85.CF.86.CE.B5.CE.BD.CF.8E.CF.83.CE.B5.CE.B9.CF.82_.CF.83.CE.B5_.CE.B4.CE.AF.CE.B3.CE.BB.CF.89.CF.83.CF.83.CE.B1_.CE.BA.CE.B5.CE.AF.CE.BC.CE.B5.CE.BD.CE.B1
% very slow, enable only at final pdf.
\usepackage[Greek,Latin]{ucharclasses}
\setTransitionsForGreek{\selectlanguage{greek}}{\selectlanguage{english}}

% polyglossia
\usepackage{polyglossia}
\setmainlanguage{greek}
\setotherlanguages{english}

% Fonts
% fonts can't go in the .fmt file
\usepackage{fontspec}
\setmainfont[Mapping=tex-text]{DejaVu Sans}
\newfontfamily\greekfont[Script=Greek]{DejaVu Sans}
\newfontfamily\greekfontsf[Script=Greek]{DejaVu Sans}
\setmonofont[Scale=1.0]{Source Code Pro Medium}
\newfontfamily\greekfonttt[Scale=1.0]{Source Code Pro Medium}
\usepackage{microtype} % microtype is font-dependant

\newcommand{\imageref}[1] {%
\hyperref[fig:#1]{Σχήμα \ref{fig:#1}}}
\newcommand{\imagerefc}[2] {%
\hyperref[fig:#1]{#2}}

\newcommand{\imagehere}[2]{%
\begin{figure}[H]%
\centering%
\includegraphics[keepaspectratio, width=1.0\linewidth]{images/#1}%
\caption{#2}
\label{fig:#1}%
\end{figure}%
}

\title{1η Εργασία στα Ψηφιακά Φίλτρα}
\author{Ορέστης Φλώρος-Μαλιβίτσης, 7796\\
Τομέας Ηλεκτρονικής,\\
Τμήμα Ηλ. Μηχανικών / Μηχανικών ΗΥ,\\
Αριστοτέλειο Πανεπιστήμιο Θεσσαλονίκης}
\titlepic{\includegraphics[width=0.40\textwidth]{../../university}}

\begin{document}
\maketitle
\tableofcontents\newpage
%\lstlistoflistings
%\listoffigures
%\listoftables
\newcommand{\weiner}[1]{\mathbf{w}\left(#1\right)}
\newcommand{\wienero}{\mathbf{w_o}}
\newcommand{\Rmatrix}{\mathbf{R}}
\newcommand{\Pvector}{\mathbf{p}}
\section{Εισαγωγή}
\imagehere{init-system}{Το σύστημά μας}
Μας δίνεται το σύστημα του \imagerefc{init-system}{Σχήματος \ref{fig:init-system}}
που περιγράφεται από τις εξής εξισώσεις:
\begin{align}
x(n) &= A\left(\sin\left(2 \pi f_o n + \phi\right) + \cos\left(4 \pi f_o n + \phi\right) + \cos\left(7 \pi n + \frac{\phi}{3} \right)\right), f_o=\frac{1}{4}, \phi=\frac{\pi}{2}, A=2.3\label{eq:x}\\
s(n) &= x(n) + v(n)\label{eq:s}\\
u(n) &= s(n - \Delta)\label{eq:u}\\
d(n) &= s(n)\\
e(n) &= d(n) - y(n)\label{eq:e}
\end{align}
Όπου το $v(n)$ είναι λευκός θόρυβος με μηδενική μέση τιμή και διακύμανση $\sigma_{v}^2 = 0.34$
και $\Delta = 10$.

\section{Ερώτημα Α}\label{section:A}
Σε διάφορες περιπτώσεις ένα ευρυζωνικό σήμα καταστρέφεται από περιοδικές παρεμβολές και δεν υπάρχει κανένα σήμα αναφοράς.
Η διάταξη του \imagerefc{init-system}{Σχήματος \ref{fig:init-system}}
μπορεί να χρησιμοποιηθεί για να απομακρύνει περιοδικές παρεμβολές που
αλλοιώνουν ένα ευρυζωνικό σήμα (broadband signal), χωρίς να απαιτεί τη χρήση σήματος αναφοράς.

Το σύνθετο σήμα $s(n)$ περιέχει το σήμα $x(n)$ και τον λευκό θόρυβο $v(n)$.
Ένα κανάλι του $s(n)$ στέλνεται στη διάταξη $Z^{-\Delta}$ που επιβάλει σταθερή καθυστέρηση $\Delta$, αρκετά μεγάλη ώστε να προκαλέσει την αποσυσχέτηση του $x(n)$ με το $v(n)$.
Το καθυστερημένο σήμα $u(n)$ φιλτράρεται μέσω του φίλτρου Wiener $\wienero$ που μας δίνει το σήμα $y(n)$ στο οποίο εμφανίζεται η περιοδική παρεμβολή. Όταν αφαιρέσουμε το $y(n)$ από το αρχικό σήμα $s(n)$ παίρνουμε το καθαρό σήμα $e(n)$.

\end{document}
