\section{Ερώτημα Α}
\subsection{Υπολογισμός πίνακα αυτοσυσχέτισης $\Rmatrix$}
Από την \eqref{eq:u} έχουμε:
\begin{align*}
&E[u(n) u^*(n-k)] = -0.78E[u(n-1) u^*(n-k)] + E[v(n) u^*(n-k)]\\
\implies{}&r_{uu}(k) + 0.78r_{uu}(k-1) = r_{vu}(k)
\numberthis
\end{align*}
όμως
\begin{equation}
r_{vu}(k) = \left\{
    \begin{array}{lr}
        \sigma_v^2,& \text{αν }k = 0\\
        0,& \text{αν }k \neq 0
    \end{array}
\right\}
\end{equation}
άρα
\begin{alignat}{3}
k = 0 &\implies r_{uu}(0) + 0.78r_{uu}(-1) = r_{vu}(0) &\implies r_{uu}(0) + 0.78r_{uu}(1) &= \sigma_v^2 = 0.19\\
k = 1 &\implies r_{uu}(1) + 0.78r_{uu}(0) = r_{vu}(1) &\implies r_{uu}(1) + 0.78r_{uu}(0) &= 0
\end{alignat}
Λύνουμε το σύστημα στο MATLAB:
\begin{lstlisting}[caption={Επίλυση συστήματος στο MATLAB}]
>> A = [1 0.78; 0.78 1];
>> B = [0.19; 0];
>> A \ B

ans =

    0.4852
   -0.3784
\end{lstlisting}
Έτσι, ο πίνακας αυτοσυσχέτισης $\Rmatrix$ είναι:
\begin{equation}
\Rmatrix = \begin{bmatrix}
    r_{uu}(0) & r_{uu}(1)\\
    r_{uu}(1) & r_{uu}(0)
\end{bmatrix} =
\begin{bmatrix}
    0.4852 & -0.3784\\
    -0.3784 & 0.4852
\end{bmatrix}
\end{equation}

\subsection{Υπολογισμός διανύσματος ετεροσυσχέτισης $\Pvector$}
Από τις \eqref{eq:u} και \eqref{eq:x+v}:
\begin{align*}
p(-k) ={}& E[u(n - k) d^*(n)]\\
={}& E[(u(n - k - 1) + v(n - k)) (x^*(n) + v^*(n))]\\
={}& -0.78E[u(n - k - 1) x^*(n)]\\
& -0.78E[u(n - k - 1) v^*(n)]\\
& + E[v(n - k) x^*(n)]\\
& + E[v(n - k) v^*(n)]
\numberthis\label{eq:p-init}
\end{align*}
Όμως επειδή το $x$ είναι ασυσχέτιστο με τα $v$ και $u$ ισχύει:
\begin{align*}
E[u(n - k - 1) x^*(n)] &= 0\\
E[v(n - k) x^*(n)] &= 0
\end{align*}
η \eqref{eq:p-init} γίνεται:
\begin{equation}
p(-k) = -0.78E[u(n - k - 1) v^*(n)] + E[v(n - k) v^*(n)] = -0.78r_{uv}(k + 1) + r_{vv}(k)
\end{equation}
Άρα, μπορούμε να βρούμε το $\Pvector$:
\begin{align*}
k = 0 \implies& p(0) = -0.78r_{uv}(1) + r_{vv}(0) = 0.19\numberthis\\
k = 1 \implies& p(-1) = -0.78r_{uv}(2) + r_{vv}(1) = 0\numberthis
\end{align*}
και
\begin{equation}
\Pvector = \begin{bmatrix} p(0) & p(-1)\end{bmatrix}^T = \begin{bmatrix} 0.19 & 0\end{bmatrix}^T
\end{equation}
