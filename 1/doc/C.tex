\section{Ερώτημα Γ}
\newcommand{\plothere}[2]{%
\begin{figure}[htbp]%
\centering%
\includegraphics[keepaspectratio, width=1.0\linewidth]{plots/#1}%
\caption{#2}
\label{fig:#1}%
\end{figure}%
}
\lstdefinestyle{chunk}
{language=Octave,
numbers=none,
aboveskip=\smallskipamount,
belowskip=\smallskipamount,
morekeywords={xcorr, audiowrite}, % missing functions
captionpos=none
}
\lstdefinestyle{declaration}
{language=Octave,
title={Declaration της συνάρτησης},
numbers=none
}
Στο αρχείο \url{matlab/steepest_descent.m} βρίσκεται η υλοποίηση του αλγορίθμου \textit{steepest descent}.
\begin{lstlisting}[style=declaration]
function [W_history, W] = steepest_descent(p, R, mu, e, max_steps)
\end{lstlisting}
Η δομή του είναι η εξής:
\begin{itemize}
\item Τα ορίσματα \lstinline!p! και \lstinline!R! παίρνουν τις τιμές $\Pvector$ και $\Rmatrix$.
\item Το όρισμα \lstinline!mu! παίρνει τις διάφορες τιμές του βήματος $\mu$.
\item Σε κάθε βήμα του αλγορίθμου η τιμή του \lstinline!W! μεταβάλλεται ως εξής:
\begin{lstlisting}[style=chunk]
W = W + mu * (p - R * W);
\end{lstlisting}
\item Η συνθήκη τερματισμού
\begin{lstlisting}[style=chunk]
diff = W_history(step) - W_history(step - 1);
if norm(diff) < e
    % shrink history size.
    W_history = W_history(1:step, :);
    break
end
\end{lstlisting}
όπου \lstinline!e! το όρισμα της συνάρτησης για την ακρίβεια.
\end{itemize}

Χρησιμοποιήθηκαν οι εξής τιμές για το $\mu$:
\begin{enumerate}
\item $\mu=0.01$: Στο \imageref{steepest1}, σύγκλιση σε 4972 βήματα.
\item $\mu=0.1$: Στο \imageref{steepest2}, σύγκλιση σε 712 βήματα.
\item $\mu=0.2$: Στο \imageref{steepest3}, σύγκλιση σε 387 βήματα.
\item $\mu=2$: Στο \imageref{steepest4}, σύγκλιση σε 47 βήματα.
\item $\mu=2.29263$: Στο \imageref{steepest5}, σύγκλιση σε 497 βήματα.
\item $\mu=2.33895$: Στο \imageref{steepest6}, απόκλιση.
\item $\mu=4$: Στο \imageref{steepest7}, απόκλιση.
\end{enumerate}

\plothere{steepest1}{Διάγραμμα προόδου της steepest descent για $\mu=0.01$}
\plothere{steepest2}{Διάγραμμα προόδου της steepest descent για $\mu=0.1$}
\plothere{steepest3}{Διάγραμμα προόδου της steepest descent για $\mu=0.2$}
\plothere{steepest4}{Διάγραμμα προόδου της steepest descent για $\mu=2$}
\plothere{steepest5}{Διάγραμμα προόδου της steepest descent για $\mu= 0.99 \mu_{max} = 2.2926$}
\plothere{steepest6}{Διάγραμμα προόδου της steepest descent για $\mu=1.01 \mu_{max} = 2.3389$}
\plothere{steepest7}{Διάγραμμα προόδου της steepest descent για $\mu=4$}

Παρατηρήθηκε ότι:
\begin{enumerate}
\item Μικρές τιμές $\mu$ συγκλίνουν αργά καθώς γίνονται πολύ μικρά βήματα.
\item Τιμές κοντά στο $\mu_{max}$ συγκλίνουν αργά καθώς αν και "φτάνουν" γρήγορα στη "περιοχή" της λύσης στο τέλος κάνουν πολύ μεγάλα βήματα και αργούν να την φτάσουν.
\item Τιμές μεγαλύτερες από το $\mu_{max}$ αποκλίνουν.
\end{enumerate}

Στα γραφήματα χρησιμοποιήθηκε ως βάση contour plot του criterion:
\[
J_w = \sigma_d^2 - 2p^Tw + w^TRw
\]
\FloatBarrier
