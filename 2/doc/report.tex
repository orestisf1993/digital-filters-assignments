%&../../preamble
% polyglossia
\usepackage{polyglossia}
\setmainlanguage{greek}
\setotherlanguages{english}

% Fonts
% fonts can't go in the .fmt file
\usepackage{fontspec,unicode-math}
\setmainfont[Mapping=tex-text]{DejaVu Sans}
\newfontfamily\greekfont[Script=Greek]{DejaVu Sans}
\newfontfamily\greekfontsf[Script=Greek]{DejaVu Sans}
\setmonofont{Hack}
\newfontfamily\greekfonttt[Scale=1.0]{Hack}
\setmathfont{Asana-Math.otf}
\setmathfont[range={\mathcal,\mathbfcal},StylisticSet=1]{Asana-Math.otf}
\usepackage{microtype} % microtype is font-dependant

\author{%
  Φλώρος-Μαλιβίτσης Ορέστης, 7796 \href{mailto:orestisf@ece.auth.gr}{orestisf@ece.auth.gr}\\
  \textbf{Τομέας Ηλεκτρονικής}\\
  \textbf{Τμήμα Ηλ. Μηχανικών / Μηχανικών ΗΥ}\\
  \textbf{Αριστοτέλειο Πανεπιστήμιο Θεσσαλονίκης}
}
\titlepic{\includegraphics[width=0.40\textwidth]{../../university}}


\renewcommand{\imageref}[1] {%
\hyperref[fig:#1]{Σχήμα \ref{fig:#1}}}
\renewcommand{\imagerefc}[2] {%
\hyperref[fig:#1]{#2}}

\newcommand{\imagehere}[2]{%
\begin{figure}[H]%
\centering%
\includegraphics[keepaspectratio, width=1.0\linewidth]{images/#1}%
\caption{#2}
\label{fig:#1}%
\end{figure}%
}

\title{2η Εργασία στα Ψηφιακά Φίλτρα\\
Προσαρμογή στο Πεδίο της Συχνότητας}
\author{Ορέστης Φλώρος-Μαλιβίτσης, 7796\\
Τομέας Ηλεκτρονικής,\\
Τμήμα Ηλ. Μηχανικών / Μηχανικών ΗΥ,\\
Αριστοτέλειο Πανεπιστήμιο Θεσσαλονίκης}
\titlepic{\includegraphics[width=0.40\textwidth]{../../university}}

\setcounter{section}{-1} % start sections from 0.
\begin{document}
\deactivateBG
\maketitle
\tableofcontents\newpage
%\lstlistoflistings
%\listoffigures
%\listoftables
\activateBG
\newcommand{\weiner}[1]{\mathbf{w}\left(#1\right)}
\newcommand{\wienero}{\mathbf{w_o}}
\newcommand{\Rmatrix}{\mathbf{R}}
\newcommand{\Pvector}{\mathbf{p}}
\section{Εισαγωγή}
\imagehere{init-system}{Το σύστημά μας}
Μας δίνεται το σύστημα του \imagerefc{init-system}{Σχήματος \ref{fig:init-system}}
που περιγράφεται από τις εξής εξισώσεις:
\begin{align}
x(n) &= A\left(\sin\left(2 \pi f_o n + \phi\right) + \cos\left(4 \pi f_o n + \phi\right) + \cos\left(7 \pi n + \frac{\phi}{3} \right)\right), f_o=\frac{1}{4}, \phi=\frac{\pi}{2}, A=2.3\label{eq:x}\\
s(n) &= x(n) + v(n)\label{eq:s}\\
u(n) &= s(n - \Delta)\label{eq:u}\\
d(n) &= s(n)\\
e(n) &= d(n) - y(n)\label{eq:e}
\end{align}
Όπου το $v(n)$ είναι λευκός θόρυβος με μηδενική μέση τιμή και διακύμανση $\sigma_{v}^2 = 0.34$
και $\Delta = 10$.

\section{Ερώτημα Α}\label{section:A}
Σε διάφορες περιπτώσεις ένα ευρυζωνικό σήμα καταστρέφεται από περιοδικές παρεμβολές και δεν υπάρχει κανένα σήμα αναφοράς.
Η διάταξη του \imagerefc{init-system}{Σχήματος \ref{fig:init-system}}
μπορεί να χρησιμοποιηθεί για να απομακρύνει περιοδικές παρεμβολές που
αλλοιώνουν ένα ευρυζωνικό σήμα (broadband signal), χωρίς να απαιτεί τη χρήση σήματος αναφοράς.

Το σύνθετο σήμα $s(n)$ περιέχει το σήμα $x(n)$ και τον λευκό θόρυβο $v(n)$.
Ένα κανάλι του $s(n)$ στέλνεται στη διάταξη $Z^{-\Delta}$ που επιβάλει σταθερή καθυστέρηση $\Delta$, αρκετά μεγάλη ώστε να προκαλέσει την αποσυσχέτηση του $x(n)$ με το $v(n)$.
Το καθυστερημένο σήμα $u(n)$ φιλτράρεται μέσω του φίλτρου Wiener $\wienero$ που μας δίνει το σήμα $y(n)$ στο οποίο εμφανίζεται η περιοδική παρεμβολή. Όταν αφαιρέσουμε το $y(n)$ από το αρχικό σήμα $s(n)$ παίρνουμε το καθαρό σήμα $e(n)$.

\section{Ερώτημα Β}
\begin{figure}[H]
\includegraphics[width=\linewidth]{plots/LMS}
\caption{Καμπύλες εκμάθησης LMS}
\label{fig:lms}
\end{figure}
\begin{figure}[H]
\includegraphics[width=\linewidth]{plots/NLMS}
\caption{Καμπύλες εκμάθησης NLMS}
\label{fig:nmls}
\end{figure}
\begin{figure}[H]
\includegraphics[width=\linewidth]{plots/RLS}
\caption{Καμπύλες εκμάθησης RLS}
\label{fig:rls}
\end{figure}

\section{Ερώτημα Γ}
\newcommand{\plothere}[2]{%
\begin{figure}[htbp]%
\centering%
\includegraphics[keepaspectratio, width=1.0\linewidth]{plots/#1}%
\caption{#2}
\label{fig:#1}%
\end{figure}%
}
\lstdefinestyle{chunk}
{language=Octave,
numbers=none,
aboveskip=\smallskipamount,
belowskip=\smallskipamount,
morekeywords={xcorr, audiowrite}, % missing functions
captionpos=none
}
\lstdefinestyle{declaration}
{language=Octave,
title={Declaration της συνάρτησης},
numbers=none
}
Στο αρχείο \url{matlab/steepest_descent.m} βρίσκεται η υλοποίηση του αλγορίθμου \textit{steepest descent}.
\begin{lstlisting}[style=declaration]
function [W_history, W] = steepest_descent(p, R, mu, e, max_steps)
\end{lstlisting}
Η δομή του είναι η εξής:
\begin{itemize}
\item Τα ορίσματα \lstinline!p! και \lstinline!R! παίρνουν τις τιμές $\Pvector$ και $\Rmatrix$.
\item Το όρισμα \lstinline!mu! παίρνει τις διάφορες τιμές του βήματος $\mu$.
\item Σε κάθε βήμα του αλγορίθμου η τιμή του \lstinline!W! μεταβάλλεται ως εξής:
\begin{lstlisting}[style=chunk]
W = W + mu * (p - R * W);
\end{lstlisting}
\item Η συνθήκη τερματισμού
\begin{lstlisting}[style=chunk]
diff = W_history(step) - W_history(step - 1);
if norm(diff) < e
    % shrink history size.
    W_history = W_history(1:step, :);
    break
end
\end{lstlisting}
όπου \lstinline!e! το όρισμα της συνάρτησης για την ακρίβεια.
\end{itemize}

Χρησιμοποιήθηκαν οι εξής τιμές για το $\mu$:
\begin{enumerate}
\item $\mu=0.01$: Στο \imageref{steepest1}, σύγκλιση σε 4972 βήματα.
\item $\mu=0.1$: Στο \imageref{steepest2}, σύγκλιση σε 712 βήματα.
\item $\mu=0.2$: Στο \imageref{steepest3}, σύγκλιση σε 387 βήματα.
\item $\mu=2$: Στο \imageref{steepest4}, σύγκλιση σε 47 βήματα.
\item $\mu=2.29263$: Στο \imageref{steepest5}, σύγκλιση σε 497 βήματα.
\item $\mu=2.33895$: Στο \imageref{steepest6}, απόκλιση.
\item $\mu=4$: Στο \imageref{steepest7}, απόκλιση.
\end{enumerate}

\plothere{steepest1}{Διάγραμμα προόδου της steepest descent για $\mu=0.01$}
\plothere{steepest2}{Διάγραμμα προόδου της steepest descent για $\mu=0.1$}
\plothere{steepest3}{Διάγραμμα προόδου της steepest descent για $\mu=0.2$}
\plothere{steepest4}{Διάγραμμα προόδου της steepest descent για $\mu=2$}
\plothere{steepest5}{Διάγραμμα προόδου της steepest descent για $\mu= 0.99 \mu_{max} = 2.2926$}
\plothere{steepest6}{Διάγραμμα προόδου της steepest descent για $\mu=1.01 \mu_{max} = 2.3389$}
\plothere{steepest7}{Διάγραμμα προόδου της steepest descent για $\mu=4$}

Παρατηρήθηκε ότι:
\begin{enumerate}
\item Μικρές τιμές $\mu$ συγκλίνουν αργά καθώς γίνονται πολύ μικρά βήματα.
\item Τιμές κοντά στο $\mu_{max}$ συγκλίνουν αργά καθώς αν και "φτάνουν" γρήγορα στη "περιοχή" της λύσης στο τέλος κάνουν πολύ μεγάλα βήματα και αργούν να την φτάσουν.
\item Τιμές μεγαλύτερες από το $\mu_{max}$ αποκλίνουν.
\end{enumerate}

Στα γραφήματα χρησιμοποιήθηκε ως βάση contour plot του criterion:
\[
J_w = \sigma_d^2 - 2p^Tw + w^TRw
\]
\FloatBarrier

\end{document}
